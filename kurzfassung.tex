\chapter*{Kurzfassung der Diplomarbeit/Abstract} \addcontentsline{toc}{chapter}{Kurzfassung der Diplomarbeit/Abstract}

Diese Diplomarbeit beschäftigt sich mit der Automatisierung von Audioanlagen auf einer Bühne, die als Proof of Concept gedacht ist, sowie einer entsprechenden Corporate Identity.

 Im Rahmen der Diplomarbeit wurde ein System entwickelt, welches die Sensoren und Servomotorensteuerung für die automatische Tonausrichtung nutzt. Die Positionsermittlung erfolgt mithilfe ESP32-UWB Sensoren. Um eine sichere Ausführung der Automatisierung zu gewährleisten wird das Mischpult in ein eigens gebautes Gerüst gestellt. Die Bühnendaten werden in einer Datenbank gespeichert. Damit die einzelnen Komponenten miteinander kommunizieren können, wird ein lokales Netzwerk verwendet.
 
Zur anschaulichen Darstellung wird in einer GUI die Echtzeitposition angezeigt.

Zur Repräsentation der Marke wurde als Werbemittel ein Flyer mit passendem Logo erstellt. 



\clearpage

\newlength{\haklogobreite}
\haklogobreite15mm
\newlength{\beschriftungsbreite}
\beschriftungsbreite124mm
\newlength{\feldA}
\feldA50mm
\newlength{\feldB}
\feldB77mm

\begin{tabular}{|p{\haklogobreite}|p{\beschriftungsbreite}|}
\hline
\multirow{2}{\haklogobreite}{  \includegraphics[width=1.0\linewidth]{images/logo-hak-noe}    }&{\vspace{0.05em}\textbf{HANDELSAKADEMIE YBBS AN DER DONAU}}\\[1.05em]
\cline{2-2}
 & { \begin{tabular}{p{\feldA} p{\feldB}}
    Fachrichtung:&\textbf{Digital Business}\\
    Ausbildungsschwerpunkte:&\textbf{Digital Business}\\
   \end{tabular}
   }\\
\hline
\end{tabular}

\begin{center}
 \LARGE \textbf{DIPLOMARBEIT}\\
 \Large \textbf{DOKUMENTATION}\\
 \normalsize
\end{center}


\newlength{\feldC}
\feldC49mm
\newlength{\feldD}
\feldD90mm

\linespread{1.1} \normalsize
\begin{tabular}{|p{\feldC}|p{\feldD}|}
 \hline
 Namen der Verfasser/innen & Michael Becksteiner, Leo Pirringer, Carina Pospichal\\ 
 \hline 
 Jahrgang & 5BK \\ Schuljahr & 2024/2025 \\
 \hline
 Thema der Diplomarbeit & Automatisierung von Tonausrichtung\\
 \hline
\end{tabular}

\begin{tabular}{|p{\feldC}|p{\feldD}|}
 \hline
 Aufgabenstellung & Erstellung eines Systems, welches die Echtzeitposition auf einer Bühne mit grafischer Oberfläche darstellt. Dazu eine passende Corporate Identity erstellen. \\
 \hline
\end{tabular}

\begin{tabular}{|p{\feldC}|p{\feldD}|}
 \hline
 Realisierung & Die Positionsermittlung erfolgte mit ESP32-UWB Sensoren und wurde mit einem JavaFX-GUI umgesetzt. Die Servomotorensteuerung erfolgt mithilfe eines gebauten Gerüsts, welches die Servomotoren in Position hält. Die Corporate Identity wurde unter Zuhilfenahme von verschiedenen Programmen wie beispielsweise Indesign und Illustrator umgesetzt.\\
 \hline
\end{tabular}

\begin{tabular}{|p{\feldC}|p{\feldD}|}
 \hline
 Ergebnisse & JavaFX-GUI. Servosteuerung und Gerüst. Eine ausgearbeitete Corporate Identity, Flyer und Logo.\\
 \hline
\end{tabular}

\begin{tabular}{|p{\feldC}|p{\feldD}|}
 \hline
 Typische Grafik, Foto etc. & \begin{minipage}{0.6\textwidth}
 	\includegraphics[width=0.8\linewidth]{images/Logo StageControl.png}
 	\label{fig:Logo StageControl_kurz}
 \end{minipage} \\
  (mit Erläuterung) & \\
 \hline
\end{tabular}

\begin{tabular}{|p{\feldC}|p{\feldD}|}
 \hline
 Möglichkeiten der Einsicht- & Archiv der Schule\\
 nahme in die Arbeit & \\
 \hline
\end{tabular}

\newlength{\feldE}
\feldE43mm

\begin{tabular}{|p{\feldC}|p{\feldE}|p{\feldE}|}
 \hline
 Approbation \small{Prüfer}& \scriptsize{Prüfer/Prüferin} & \scriptsize{Direktor bzw. Abteilungsvorstand}\\ 
 (Datum / Unterschrift)& & \\
 \hline
\end{tabular}
\linespread{1.25} \normalsize

\clearpage

% Makro um in vorgegebener Spaltenbreite zentrieren zu können
\newcolumntype{C}[1]{>{\centering\arraybackslash}m{#1}}

\begin{tabular}{|p{\haklogobreite}|p{\beschriftungsbreite}|}
	\hline
	\multirow{2}{\haklogobreite}{  \includegraphics[width=1.0\linewidth]{images/logo-hak-noe}    }&{\vspace{0.15em}\textbf{HANDELSAKADEMIE YBBS AN DER DONAU}}\\[1.05em]
	\cline{2-2}
	& { \begin{tabular}{p{\feldA} p{\feldB}}
    		Department:&\textbf{Digital Business}\\
			Educational focus:&\textbf{Digital Business}\\
		\end{tabular}
	}\\
	\hline
\end{tabular}

\begin{center}
 \LARGE \textbf{DIPLOMA THESIS}\\
 \Large \textbf{Documentation}\\
 \normalsize
\end{center}

\linespread{1.1} \normalsize
\begin{tabular}{|p{\feldC}|p{\feldD}|}
 \hline
 Author(s) & Michael Becksteiner, Leo Pirringer, Carina Pospichal\\
 \hline
 Form & 5BK \\ Academic year & 2024/2025 \\
 \hline
 Topic & Automation of sound alignment \\
 \hline
\end{tabular}

\begin{tabular}{|p{\feldC}|p{\feldD}|}
 \hline
 Assignment of Tasks & Creation of a system that displays the real-time position on a stage with a graphical interface. Create a suitable corporate identity for this.\\
 \hline
\end{tabular}

\begin{tabular}{|p{\feldC}|p{\feldD}|}
 \hline
 Realisation & The position was determined using ESP32-UWB sensors and implemented with a JavaFX GUI. The servomotors are controlled with the help of a built frame that holds the servomotors in position. The corporate identity was implemented using various programs such as Indesign and Illustrator \\
 \hline
\end{tabular}

\begin{tabular}{|p{\feldC}|p{\feldD}|}
 \hline
 Results & JavaFX GUI. Servo control and scaffolding. An elaborated corporate identity, flyer and logo. \\
 \hline
\end{tabular}

\begin{tabular}{|p{\feldC}|p{\feldD}|}
 \hline
 Illustrative Graph, Photo & \begin{minipage}{0.6\textwidth}
 	\includegraphics[width=0.8\linewidth]{images/Logo StageControl.png}
 	\label{fig:Logo StageControl_short}
 \end{minipage} \ \\
 (incl. explanation) & \\
 \hline
\end{tabular}

\begin{tabular}{|p{\feldC}|p{\feldD}|}
 \hline
 Accessibility of Diploma Thesis & Archive of the school \\
 \hline
\end{tabular}

\begin{tabular}{|p{\feldC}|p{\feldE}|p{\feldE}|}
 \hline
 Approval & \scriptsize{Examiner} & \scriptsize{Head of College / Department}\\ 
 (Date / Sign)& & \\
 \hline
\end{tabular}
\linespread{1.25} \normalsize
