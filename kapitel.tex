\chapter{Zielsetzung und Aufgabenstellung}

Individuelle Zielsetzung und Aufgabenstellung mit Terminplan der einzelnen Teammitglieder.

\chapter{Grundlagen und Methoden}

Im \textbf{Kapitel 3} werden die \textbf{Grundlagen und Methoden} geklärt.

\begin{itemize}
 \item IST-Situation
 \item Lösungsansätze
 \item Begründung der gewählten Methodik
\end{itemize}

Nutze Bilder inkl. Beschriftung, Tabellen und Boxen zur Inhaltsstrukturierung. Alle nötigen
Formate und Stile stehen bereit. Die Abbildungen werden automatisch fortlaufend
nummeriert.


\begin{figure}[!ht]
 \begin{center}
  \epsfig{figure=images/ECG8e_board_2.eps, width=10cm, angle=0}
  \caption{Grafiken immer beschriften.}
  \label{Verweisbezeichnung der Grafik}
 \end{center}
\end{figure}

\section{Hinweise für Neulinge}

Compiliert werden muss in diesem Archiv nur das File main.tex.

Empfohlen für Linux:
\begin{itemize}
 \item Editor: kile
 \item Pakete: texlive-full
 \item Viewer: okular
 \end{itemize}

Empfohlene Windows Pakete, in dieser Reihenfolge installieren:
\begin{itemize}
 \item GSView
 \item Ghostscript
 \item Miktex
 \item Editoren: WinEdt oder TCXSetup oder LEdBeta oder TeXaide
\end{itemize}

Wenn möglich sollten Vektorgrafiken im eps Format eingebunden werden.

Der Ablauf zur Erzeugung eines hochwertigen Dokumentes gestaltet sich folgendermaßen:

% Makro um in vorgegebener Spaltenbreite zentrieren zu können
%\newcolumntype{C}[1]{>{\centering\arraybackslash}m{#1}}

\begin{center}

 \begin{tabular}{|C{50mm}|C{30mm}|p{50mm}|}
  \hline
  Quellcode & main.tex & z.B. in kile geschrieben\\
  \hline
 \end{tabular}
 
 $\downarrow$
 
  \LaTeX \\
 
 $\downarrow$
 
 \begin{tabular}{|C{50mm}|C{30mm}|p{50mm}|}
  \hline
  device independent file format & main.dvi & Vorschau de facto in Echtzeit z.B. in okular verfügbar\\
  \hline
 \end{tabular}
 
 $\downarrow$
 
 dvips
 
 $\downarrow$
 
 \begin{tabular}{|C{50mm}|C{30mm}|p{50mm}|}
  \hline
  post script file format & main.ps & hohe Qualität, Bilder unkomprimiert, kann direkt gedruckt werden, Ansicht z.B. in okular\\
  \hline
 \end{tabular}
 
 $\downarrow$
 
 ps2pdf
 
 $\downarrow$
 
  \begin{tabular}{|C{50mm}|C{30mm}|p{50mm}|}
  \hline
  portable document format & main.pdf & verlustbehaftet komprimiert\\
  \hline
 \end{tabular}
 
\end{center}

Es sollte also während des Editierens nur der \LaTeX - Button gedrückt werden und daneben die aktuelle .dvi Vorschau betrachtet werden.
Nach Abschluss der Setzarbeit kann der rechenaufwändige vollständige Ablauf beschritten werden.
In kile kann dazu in den Settings zuvor der QuickBuild-Button mit der empfohlenen Sequenz belegt werden.

\section{Vertiefendes Wissen:}

\begin{itemize}
 \item Grundkurs \url{https://portal.sz-ybbs.ac.at/%7Eszauner/Latex/latex_basic.pdf}
 \item Fortgeschrittenenkurs \url{https://portal.sz-ybbs.ac.at/%7Eszauner/Latex/latex_advanced.pdf}
 \item Allgemeiner Kurs \url{https://portal.sz-ybbs.ac.at/%7Eszauner/Latex/latex-kurs.pdf}
 \item Symbolliste \url{https://portal.sz-ybbs.ac.at/%7Eszauner/Latex/symbols-a4.pdf}
 \item Das kleine TeX Buch: \url{https://portal.sz-ybbs.ac.at/%7Eszauner/Latex/TeXbuch.pdf}
\end{itemize}



\chapter{Ergebnisdokumentation}

Versuche nun, eine neue Seite mit automatischem Textfluss zu erzeugen.

\section{Hilfsformate}

\begin{itemize}
 \item Hilfsformat Brüche: 1/4 = $\frac{1}{4}$
 \item Markierungen: \color{black}\colorbox{red}{rote Markierung}
\end{itemize}


\textcolor{red}{Farbiger Text.}


\chapter*{Quellen und Literatur}

Bibliography

\textbf{Nachname, Vorname: Titel. Untertitel. - Verlagsort: Verlag, Jahr.}

\textbf{Nachname, Vorname} (Herausgeber): Titel. Untertitel. Auflage - Verlagsort: Verlag, Jahr.

\textbf{Nachname, Vorname des Autors des bearbeiteten Artikels: Titel des Artikels. In: Titel der
Zeitschrift, Heftnummer, Jahrgang, Seite} (eventuell: Verlagsort, Verlag).

\textbf{Nachname, Vorname des Autors: Titel. Online in Internet: URL: www-Adresse, Datum.}\\
(Autor und Titel wenn vorhanden, Online in Internet: URL: www-Adresse, Datum auf jeden Fall)


