
\pagestyle{fancy} \frenchspacing
\lhead{StageControl - automatisierte Steuerung von Ton- und Lichtanlagen (2024/2025)}
\lfoot{Name 1}
\renewcommand{\chaptermark}[1]{\markboth{#1}{}}

\renewcommand{\textfraction}{0}
\renewcommand{\floatpagefraction}{0.999}
\renewcommand{\topfraction}{0.7}
\renewcommand{\bottomfraction}{0.999}

\chapter{Grundlagen und Methoden}

\section{Etablierte Lösungsansätze}
Das Kapitel listet etablierte Lösungsansätze zur Standortermittlung einer Person auf einer Bühne und erklärt diese genauer. Ebenso wie Vor- und Nachteile anhand eines Beispieles.

\subsection{Ausgangsituation des Praxisbeispiels}

Man befindet sich auf der Bühne in der Stadthalle in Ybbs. Folgende Informationen sind wichtig.
\begin{itemize}
	\item \textbf{Gerät zur Standortermittlung: } Android Smartphone
	\item \textbf{Koordinaten der Position: } TBD  
\end{itemize}

\subsection{Manuelle Steuerung}
Eine Person steht auf der Bühne der XY. Eine Möglichkeit die Steuerung der Ton- und Lichtanlagen ist diese für das Event vorher zu programmieren oder manuell während der Show zu steuern. Nun fragt sich die Person: "Wie kann ich diese Ton- und Lichtanlagen steuern?" Die Antworten folgen:

\begin{itemize}
	\item "Manuelle Steuerung der Tonanlage"
	\item "Vorprogrammierung der Lichtanlage"
\end{itemize}

An den erlangten Antworten, kann man erkennen, dass es noch keine automatisierte Lösung für das Problem gibt. Die Genauigkeit der Standortermittlung, die für die automatisierte Steuerung der Anlagen notwendig ist, kann in folgenden Stufen eingeteilt werden: 

\begin{itemize}
	\item \textbf{Stufe 1: }Standort auf Bühne eingeschränkt
	\item \textbf{Stufe 2: }Standort auf Länge und Breite der Bühne eingeschränkt
	\item \textbf{Stufe 3: }Standort auf bestimmten Punkt auf vorhin genannter Bühne eingeschränkt
\end{itemize}

\section{Hardware im Zusammenhang mit StageControl}
Im Fall von StageControl versteht man unter Hardware nicht nur Computerkomponenten, sondern auch hergestellte Konstruktionen, die das Funktionieren der Software bzw. der angesteuerten Komponenten erst möglich machen.

\subsection{Möglichkeiten der Hardwareproduktion}
Die Hardwareproduktion umfasst die Herstellung physischer Komponenten aus unterschiedlichsten Materialien wie z. B. PLA (Polylactide, ein Kunststoff) oder Metallen. Diese Stoffe werden von Unternehmen eingesetzt, die sich mit der Herstellung von Hardware bzw. Prototypen befassen. Dabei gibt es verschiedenste Möglichkeiten und Herangehensweisen in der Hardwareproduktion: 3D-Druck, Laser-Cutting, CNC-Fräsen und die Verwendung von Baukästen wie LEGO®. Im folgenden Abschnitt wird genauer auf die Einsetzbarkeit, die Vor- und Nachteile sowie den Nutzen dieser Produktionsarten für StageControl eingegangen.

\subsection{3D-Druck}
Beim 3D-Druck handelt es sich um ein Verfahren, das Schicht für Schicht Material aufträgt, um ein dreidimensionales Objekt zu erschaffen. Diese Art der Produktion wird auch als „Additive Fertigung“ bezeichnet, im Gegensatz zum CNC-Fräsen, das als „subtraktive Fertigung“ bekannt ist.

Die wichtigsten und gängigsten Drucktechnologien im Überblick:

\begin{itemize}
	\item \textbf{FFF:} (Fused Filament Fabrication) oder \textbf{FDM} (Fused Deposition Modelling): Hierbei wird Kunststofffilament \emph{(eine einzelne Faser beliebig lang)} verwendet.
	\item \textbf{SLA:} Diese Technologie verwendet lichtempfindliches Harz, das sich verfestigt.
	\item \textbf{PBF:} Eine Möglichkeit, diese pulverbasierte Technologie zu nutzen, ist, das Pulver mittels Laser miteinander zu verschmelzen. 
\end{itemize}

\subsection{Vorteiler des 3D-Drucks}

\begin{itemize}
	\item Möglichkeit, komplexe Objekte herzustellen.
	\item Keine Vorlaufzeit nötig, d.h. keine Werkzeugproduktion erforderlich.
	\item Rapid Prototyping \emph{(deutsch „schnelle Prototypenherstellung“)}
\end{itemize}

\subsection{Technische Voraussetzungen}
Um den 3D-Druck technisch durchführen zu können, werden folgende Komponenten benötigt:
\begin{itemize} 
\item 3D-Drucker 
\item 3D-Modellierungsprogramm
\item 3D-Druck Material (Filament)
\item Computer
\end{itemize}
Auf diese Punkte wird in auf den nächsten Seiten näher eingegangen.

\section{3D-Drucker}
Work in progress...


\section{3D-Modellierungsprogramme}
Auf dem Markt...


According to the report, businesses face significant damages due to cyber crime as depicted by \emph{Table 1} below: 

According to \textcite{embroker}, $\frac{2}{3}$ of companies...

$\frac{2}{3}$ of all companies experienced a cyberattack within the past 4 years \parencite{embroker}

\begin{table} [H]
	\begin{tabular}{ |p{2cm}|p{11.0cm}| }
		\hline
		\textbf{Percent Affected}& \textbf{Issues Experienced}\\
		\hline
		45 \% & Report their countermeasures are inefficient  \\ 
		66 \% & Report they have been affected by a cyber attack in the previous year  \\  
		69 \% & Claim the attacks become increasingly targeted \\  
		\hline
	\end{tabular}
	\caption{\label{tab:CyberSecReport}Ponemon Institute State of Cybersecurity Report Statistics.}
\end{table}

.... more text ....  (see \emph{Table 2}):

\begin{table} [H]
	\begin{tabular}{ |p{2cm}| p{11.0cm}| }
		\hline
		\textbf{\#}& \textbf{Security Risk}\\
		\hline
		1 & Broken Access Control \\
		2 & Cryptographic Failures \\
		3 & Injection (Database) \\
		\hline
	\end{tabular}
	\caption{\label{tab:OWASP2021} OWASP Top 10 Web Application Security Risks 2021}
\end{table}

\begin{lstlisting}
System.out.printlin("Some code..." );
\end{lstlisting}



A sample figure: \\

\begin{figure}[H]
	\centering
	\includegraphics[width=0.5\linewidth]{images/SZ-Ybbs-Logo}
	\caption[Short Description]{Short Description}
	\label{fig:sz-ybbs-logo}
\end{figure}

\chapter{Ergebnisdokumentation}
\lfoot{Name 2}

Sample Table without label:

\begin{itemize}
	\item 	Item 1
	\item 	Item 2
\end{itemize}

Sample Math Formula: \\

$4 + 4 = ?$ \\


Text citation sample
\textcite{Vigna}. \\


\textbf{Bold Text Example}

Another text citation \textcite{OWASP},  \\

( ...  ) \\

Parencite Example \parencite{embroker}. \\

\textbf{Constructivism}

( ...  ) Third textcite \textcite{DeciRyan} \emph{Table 3}:

\begin{table} [H]
	\begin{tabular}{ |p{4cm}| p{9.0cm}| }
		\hline
		\textbf{Model}& \textbf{Characteristics}\\
		\hline
		Behaviorism & Wanted output is triggered by an appropriate input. Authoritarian model; the teacher knows what to do and shall find ways to impose his ideas on the student. \\
		Cognitivism & The learner solves given tasks on their own The teacher attends the learning process, observes, where necessary provides aid. \\
		Constructivism & Focus on personal experience. The learner is tasked with solving complex problems and have to find solutions on their own. The teacher fulfills a 	role akin to a coach. The Teacher provides aid by means of his experience and expertise in solving complex problems. \\
		\hline
	\end{tabular}
	\caption{\label{tab:learningModelComparison} Comparing Lerning Models}
\end{table}

\begin{lstlisting}
	System.console.write("Hello world");
\end{lstlisting}

( ... )




