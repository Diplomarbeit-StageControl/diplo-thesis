
\pagestyle{fancy} \frenchspacing
\lhead{StageControl - automatisierte Steuerung von Ton- und Lichtanlagen (2024/2025)}
\lfoot{}
\renewcommand{\chaptermark}[1]{\markboth{#1}{}}

\renewcommand{\textfraction}{0}
\renewcommand{\floatpagefraction}{0.999}
\renewcommand{\topfraction}{0.7}
\renewcommand{\bottomfraction}{0.999}

\chapter{Grundlagen und Methoden}

\section{Etablierte Lösungsansätze}
Das Kapitel listet etablierte Lösungsansätze zur Standortermittlung einer Person auf einer Bühne und erklärt diese genauer. Ebenso wie Vor- und Nachteile anhand eines Beispieles.

\subsection{Ausgangsituation des Praxisbeispiels}

Man befindet sich auf der Bühne in der Stadthalle in Ybbs. Folgende Informationen sind wichtig.
\begin{itemize}
	\item \textbf{Gerät zur Standortermittlung: } Android Smartphone
	\item \textbf{Koordinaten der Position: } TBD  
\end{itemize}

\subsection{Manuelle Steuerung}
Eine Person steht auf der Bühne der XY. Eine Möglichkeit die Steuerung der Ton- und Lichtanlagen ist diese für das Event vorher zu programmieren oder manuell während der Show zu steuern. Nun fragt sich die Person: "Wie kann ich diese Ton- und Lichtanlagen steuern?" Die Antworten folgen:

\begin{itemize}
	\item "Manuelle Steuerung der Tonanlage"
	\item "Vorprogrammierung der Lichtanlage"
\end{itemize}

An den erlangten Antworten, kann man erkennen, dass es noch keine automatisierte Lösung für das Problem gibt. Die Genauigkeit der Standortermittlung, die für die automatisierte Steuerung der Anlagen notwendig ist, kann in folgenden Stufen eingeteilt werden: 

\begin{itemize}
	\item \textbf{Stufe 1: }Standort auf Bühne eingeschränkt
	\item \textbf{Stufe 2: }Standort auf Länge und Breite der Bühne eingeschränkt
	\item \textbf{Stufe 3: }Standort auf bestimmten Punkt auf vorhin genannter Bühne eingeschränkt
\end{itemize}

\section{Hardwareproduktion}
Im Fall von StageControl versteht man unter Hardware nicht nur Computerkomponenten, sondern auch speziell hergestellte Konstruktionen, die das Funktionieren der Software bzw. der angesteuerten Komponenten erst ermöglichen. So benötigen wir eine Konstruktion, die es ermöglicht, die Servomotoren, die die Schieberegler steuern um den gewünschten Stereoeffekt erzeugen, möglichst einfach zu montieren. 

\subsection{Möglichkeiten der Hardwareproduktion}
Die Hardwareproduktion umfasst die Herstellung physischer Komponenten aus unterschiedlichsten Materialien wie z. B. Kunststoff, Metall oder Holz. Diese Stoffe werden von Unternehmen eingesetzt, die sich mit der Herstellung von Hardware bzw. Prototypen befassen. Dabei gibt es verschiedenste Möglichkeiten und Herangehensweisen in der Hardwareproduktion: 3D-Druck, Laser-Cutting, CNC-Fräsen und die Verwendung von Baukästen wie LEGO®. Im folgenden Abschnitt wird genauer auf die Einsetzbarkeit/Verfügbarkeit, die Vor- und Nachteile zweier Produktionsarten, nämlich 3D-Druck und  für StageControl eingegangen.

\section{3D-Druck}
\subsection{Wie funktioniert 3D-Druck}
Beim 3D-Druck handelt es sich um ein Verfahren, das Schicht für Schicht Material aufträgt, um ein dreidimensionales Objekt zu erschaffen. Dabei wird aus einer Düse das heiße Material aufgetragen, bis aus vielen Schichten das gewünschte Objekt erstellt wurde. Diese Art der Produktion wird auch als „Additive Fertigung“ bezeichnet, im Gegensatz zum CNC-Fräsen, das als „subtraktive Fertigung“ bekannt ist.

\subsection{Vorteile des 3D-Drucks}
3D-Druck bietet viele Vorteile für Privatkunden und Unternehmen. Die nachfolgenden Vorteile zählen zu den wichtigsten:
\begin{itemize}
	\item Möglichkeit, komplexe Objekte relativ schnell herzustellen.
	\item Keine Vorlaufzeit nötig, d.h. keine Werkzeugproduktion erforderlich.
	\item Rapid Prototyping \emph{(deutsch „schnelle Prototypenherstellung“)}
	\item Kostengünstige Produktion
	\item Herstellung induvidueller Objekte
\end{itemize} \cite{3D-Druck-Vorteile}

\subsection{Nachteile des 3D-Druck}
3D-Druck bietet aber nicht nur Vorteile, sondern hat wie alle Technologien seine Nachteile und Schwächen.
\begin{itemize}
	\item Nachbearbeitung der Konstruktionen nötig.
	\item Nicht so genau wie subtraktive Fertigungverfahren.
	\item Lange Fertigungszeiten
	\item Begrenztes Bauvolumen
\end{itemize}
\cite{3D-Druck-Nachteile}

\subsection{Technische Voraussetzungen}
Um den 3D-Druck technisch durchführen zu können, werden folgende Komponenten benötigt:
\begin{itemize} 
	\item 3D-Drucker
	\item Slicer
	\item CAD-Programm
	\item 3D-Druck Material (Filament)
	\item Computer
\end{itemize}
Auf den nächsten Seiten wird auf diese Punkte näher eingegangen und ein Vergleich erstellt um die Vor- und Nachteile der genannten 3D-Drucker zu veranschaulichen. \cite{3ds}


\subsection{Druckverfahren}

Um Hardware mittels 3D-Drucks zu erstellen, gibt es verschiedene Möglichkeiten, das Ausgangsmaterial in die gewünschte Form zu bringen. Die gängigsten und weitverbreitesten Verfahren im Überblick: 

\begin{itemize}
	\item \textbf{FFF:} (Fused Filament Fabrication) oder \textbf{FDM} (Fused Deposition Modelling): Hierbei wird Kunststofffilament \emph{(eine einzelne Faser beliebig lang)} verwendet und Schicht für Schicht aufgetragen.
	\item \textbf{SLA:} Diese Technologie verwendet lichtempfindliches Kunstharz, das sich verfestigt. Wird ebenfalls Schicht für Schicht aufgetragen.
	\item \textbf{PBF:} Eine Möglichkeit, diese pulverbasierte Technologie zu nutzen, ist, das Pulver mittels Laser miteinander zu verschmelzen. \cite{kaffka} \cite{3ds}
\end{itemize}

\subsection{Werkstoffe}
3D-Drucker benötigen zum Drucken eines Objektes Material, das sie verbrauchen können, um das Objekt herzustellen. Im Rahmen dieser Diploarbeit werden 2 Werkstoffe ganuer analysiert, nämlich Nylon und Kohlefaser.

\subsubsection{Nylon}
Nylon \emph{(techn. Polyamid)} ist ein langlebiges Material, das sich vorallem durch seine Widerstandfähigkeit gegen Hitze und mechanischen Auseneinwirkungen auszeichnet. Nylon gibt es in unterschiedlichen Ausführung zb. als Filament, Draht oder Pulver mit unterschiedlichen eigenschaften. \cite{Nylon}
\subsubsection{Kohlefaser}
Kohlefasern wird in Grundpolymeren wie PLA oder Nylon eingearbeitet um eine höhere Stabilität und Resistenz zu erzeugen wie zb. Nylon verstärkt mit Kohlefasern. \cite{Kohlefasern}


\subsection{3D-Drucker}

\subsubsection{Vergleich der 3D-Drucker}
Wie in der Einleitung des Kapitels \emph{"Hardware im Zusammenhang mit StageControl"} beschrieben, ist eine Möglichkeit zur Herstellung der benötigten Hardware das Drucken mittels eines 3D-Druckers. Um die bereits erwähnten 3D-Drucker (\emph{Bambu Lab X1-Carbon, Ultimaker 2 Extended+}) miteinander zu vergleichen, werden folgende Kriterien herangezogen: die Verarbeitbarkeit des Ausgangsmaterials und die Druckgeschwindigkeit bei einem Düsendurchmesser von 0,4 mm. 

Als Ausgangsmaterial wird bei beiden 3D-Druckern kohlefaserverstärktes Nylon-Filament verwendet.

\begin{table} [H]
	\begin{tabular}{ |p{2.7cm} |p{4.8cm}|p{4.8cm}| }
		\hline
		 \textbf{Kriterien} & \textbf{Ultimaker 2 Extended+}& \textbf{Bambu Lab X1-Carbon 3 Pro}\\
		\hline
		\textbf{Material} & Material kann verarbeitet werden & Materiel kann verarbeitet werden   \\ 
		\hline
		\textbf{Drucktempo} & Druckt mit einer Geschwindigkeit von 16mm\textsuperscript{3}/s & Druckt mit einer Geschwindigkeit von 32mm\textsuperscript{3}/s   \\  
		\hline
		\textbf{Druckvolumen} & 223 x 223 x 205 mm & 256 x 256 x 256 mm \\
		\hline
		\textbf{Betriebssysteme} & MacOS, Windows, and Linux & Windows, MacOS \\
		\hline
	\end{tabular}
	\caption{Vergleich  Ultimaker 2 Extended+ und Bambu Lab X1-Carbon 3 Pro}
\end{table}
\cite{BambuLabX1-Carbon3DPrinter-Specifications}, \cite{Ultimaker2Extended+-Specification}

\subsection{Slicer}
\subsubsection{Was ist ein Slicer?}
Ein Slicer ist ein Stück Software, die es erst möglich macht 3D-Modelle auszudrucken, indem es sogenannten G-Code erstellt. G-Code ist die Programmiersprache die von Computern verwendet wird um mit Maschinen zu kommunizieren, wenn die Maschine Bewegungnen ausführen soll. 
\cite{Slicer_G-Code}


\section{CNC-Fräsen}

Ein weiterer Ansatz zur Hardware- bzw. Prototypenherstellung, der für StageControl benötigt wird, ist das CNC-Fräsen (\emph{engl. "Computerized Numerical Control"}). Diese Technologie wird vor allem in der metallverarbeitenden Industrie angewendet, da sie äußerst genau und kostengünstig arbeitet. Zudem beschränkt sich das CNC-Fräsen nicht nur auf Metall; es kann auch bei der Bearbeitung von Kunststoffen und Holz eingesetzt werden. Anders als bei der 3D-Druck-Technologie handelt es sich hier nicht um ein additives, sondern um ein subtraktives Verfahren \cite{CNC-Fräsen} \cite{CNC-Fräsen_2} \cite{CNC-Fräsen_3}.

\subsection{Wie funktioniert CNC-Fräsen?}
Wie bereits in der Einleitung zum CNC-Fräsen erläutert, handelt es sich um ein subtraktives Verfahren, bei dem Material entfernt wird. Das Ausgangsmaterial wird fest in die Maschine eingespannt, während sich der Fräser bewegt. Der Fräser trägt dann das Material ab, um die gewünschte Form herzustellen. Um besonders präzise Abrundungen zu erzeugen, wird nicht das Material, sondern der Fräskopf fixiert, während sich das Material bewegt. Wie beim 3D-Druck wird hier auch G-Code als Steuerung der Maschine eingestzt. \cite{CNC-Fräsen_2} \cite{CNC-Fräsen_3}.

\subsubsection{Einsatzmöglichkeiten}
CNC-Fräsmaschinen können vielseitig eingesetzt werden und eine Vielzahl von Objekten in unterschiedlichsten Größen und Formen herstellen. Zu den typischen Anwendungsbereichen zählen unter anderem:

\begin{itemize}
	\item Maschinenteilherstellung: Komponenten von Maschinen, die nur geringe Toleranzen zulassen.
	\item Prototypen: Herstellung von ersten Modellen in geringen Stückzahlen.
\end{itemize}

Durch den Einsatz von CNC-Fräsen können menschliche Fehler und die Ausschussrate minimiert werden. Dies führt zu niedrigeren Stückkosten, was sich besonders bei der Serienfertigung bemerkbar macht. \cite{CNC-Fräsen_2} \cite{CNC-Fräsen_3}.

\subsubsection{Vorteile des CNC-Fräsens}
CNC-Fräsen bietet im Vergleich zum herkömmlichen, manuellen Fräsen einige Vorteile. Wie bei jeder manuellen Arbeit können Arbeitsunfälle passieren. Durch den Einsatz moderner CNC-Fräsen, die aus der Ferne gesteuert werden können, lassen sich solche Unfälle jedoch vermeiden. Ein weiterer Vorteil des CNC-Fräsens ist seine gleichbleibende Genauigkeit, da die CNC-Fräse computergesteuert ist. Diese Technologie wird vor allem bei Werkstücken eingesetzt, die nur wenig Toleranz in der Genauigkeit zulassen. CNC-Fräsen arbeiten mit einer Präzision von 0,01 bis 0,03 mm.
\cite{CNC-Fräsen_Vorteile}

\subsection{CNC gesteuerte Maschinen}
Es gibt nicht nur CNC-Fräser oder Dreher, sondern alle Maschinen die durch CNC (\emph{rechnergestützte numerische Steuerung}) gesteuerten werden. Durch den Einsatz von CNC lassen sich die verschiedensten Maschinen Computergestützt steuern. Zu diesen Maschinen zählen unteranderem CNC-Bohrmaschinen,  CNC-Laserschneidmaschinen, CNC-Schleifmaschinen, CNC-Wasserstrahlschneidmaschinen oder der 3D-Drucker.
\cite{Arten_CNC_Maschinen}

\subsection{Unterschied zwischen CNC-Fräsen und CNC-Drehen}
Beim CNC-Drehen wird ein Spannfutter anstatt des Schneidewerkzeugs eingesetzt. Dies ermöglicht die Herstellung von runden, zylindrische oder konische Formen. Mit diesem Verfahren lassen sich keine Objekte herstellen die eine hohe Präzision erfordern. Im Gegensatz zur CNC-Fräsmaschine arbeitet das CNC-Drehen nur auf zwei Achsen. Zudem dreht sich das Material und der Fräser ist fixiert, so wird das Material wie bei der CNC-Fräse abgetragen.
\cite{CNC-Drehen_Unterschied}


\subsection{Werkstoffe}
Bei der CNC-Fräsung können die unterschiedlichsten Materialien verwendet werden. Materialien, die eine CNC-Fräse verarbeiten kann, sind zerspanbar. Zu den gängigsten zählen:

\begin{itemize}
	\item Metalle, z. B. Stahl, Aluminium, Titan, Bronze, Messing.
	\item Holz.
	\item Zerspanbare Kunststoffe, z. B. PEI (\emph{Polyetherimid}), die sich durch ihre mechanische Festigkeit und Steifigkeit auszeichnen.
\end{itemize}
\cite{CNC-Fräsen_3} \cite{PEIZerspannung} \cite{PEIKunststoff-Polyetherimid}

\subsection{Fertigungsmöglichkeiten}

\subsubsection{3-Achs-Fräsung}
Hierbei kann sich der Fräser auf drei Achsen bewegen, nämlich auf der X-, Y- und Z-Achse. So kann das Material von allen drei Seiten bearbeitet werden. Es handelt sich um das einfachste Fräsverfahren. Zwar können auch komplexere Objekte erstellt werden, dies erfordert jedoch ein mehrfaches Umspannen des Werkstücks.

\subsubsection{4-Achs-Fräsung}
Zusätzlich zu den bereits bekannten drei Achsen kann nun eine vierte Achse integriert werden, nämlich das Schwenken der Aufspannung des Frästeils. Dadurch kann das Umspannen, das bei der 3-Achs-Fräsung notwendig ist, vermieden werden.

\subsubsection{5-Achs-Fräsung}
Bei der 5-Achs-Fräsung kann nicht nur die Aufspannung des Frästeils zum Schwenken gebracht werden sonder der Maschinentisch selbst ist rotierbar. So kann das Werkstück von 5 Seiten in einer Aufspannung bearbeitet werden. Diese Fräsen werden eingestzt bei der Produktion höchstkomplexer Strukturen.

\cite{Fräsen-3/4/5-Achs}


\section{3D-Modellierungsprogramme}
Auf dem Markt rund um C3D-Modellierungsprogramme gibt es eine große Auswahl an den verschiedensten Programmen. Von dem anfängerfreundlichen Tinkercad, das durch seine einfache Bedienbarkeit und zahlreiche Tutorials überzeugt, bis hin zu einem der führenden CAD-Programme wie Catia, das in der Industrie für seine leistungsstarken und umfangreichen Funktionen geschätzt wird, ist alles dabei. Im Rahmen dieser Diplomarbeit wurden einige der Marktführer recherchiert und miteinander verglichen.
\cite{CAD-Programme}, \cite{3D-Printing-Software}

\subsection{Vergleich verschiedener CAD-Programme}

\subsubsection{Autodesk Fusion}
Es eröffnet zahlreiche Anwendungsmöglichkeiten, die von der Prototypenerstellung bis hin zur Entwicklung von Konsumgütern reichen. Dieses Programm bietet umfangreiche Werkzeuge für Design, Engineering und Fertigung. Zudem wird Fusion 360 von vielen internationalen und nationalen Unternehmen verwendet, weil es vielseitig einsetzbar ist und sowohl in kleinen als auch in großen Projekten hervorragende Unterstützung bietet. Dank seiner cloudbasierten Plattform ermöglicht Fusion 360 eine nahtlose Zusammenarbeit und den einfachen Austausch von Projektdaten, was es zu einer bevorzugten Wahl in der modernen Produktentwicklung macht. 
\cite{AutodeskFusion}

\textbf{Vorteile}
\begin{itemize}
	\item Genutzt von vielen Namhaften Unternehmen wie zb. Yamaha oder Panasonic
	\item Große Community
	\item Kostenlose Version für Schulen
\end{itemize} 

\textbf{Nachteile}
\begin{itemize}
	\item Cloud Based; Kann zu Problemen führen, wenn offline genutzt 
	\item Benötigt eine schnelle Internetverbindung
	\item Hoher Anspruch an die Hardware des Computers
\end{itemize}
\cite{AutodeskFusionReviews}


\subsubsection{Blender} 
Blender bietet vielseitige Anwendungsmöglichkeiten, von dellierung und Animation bis hin zur Spieleentwicklung und visuellen Effekten. Dieses Open-Source-Programm bietet umfassende Werkzeuge für Modellierung, Simulation, Rendering, Compositing und Motion Tracking. Zudem wird Blender von vielen internationalen und nationalen Unternehmen sowie unabhängigen Kunden verwendet, weil es viele Einsatzmöglichkeiten bietet und Gratis zugänglich für jeden ist. Dank seiner aktiven Community und regelmäßigen Updates bleibt Blender immer auf dem neuesten Stand der Technik. 
\cite{Blender}

\textbf{Vorteile}
\begin{itemize}
	\item Gratis 
	\item Open Source Software
	\item Funktioniert auf allen gängigen Betriebssystemen
	\item Regelmäßige Updates
	\item große Community
\end{itemize}

\textbf{Nachteile}
\begin{itemize}
	\item Kein Industriestandard - wird nicht von großen Unternehmen genutzt
	\item Nicht die Benutzerfreundlichste Oberfläche
\end{itemize}
\cite{BlenderPros&Cons}


\subsubsection{FreeCAD}
FreeCAD hat vielseitige Anwendungsmöglichkeiten, von der dellierung bis hin zur Erstellung technischer Zeichnungen und Simulationen. Diese Open-Source-Software bietet umfassende Werkzeuge für Ingenieure, Architekten, Produktdesigner und Privatkunden. Zudem wird FreeCAD von vielen Unternehmen sowie unabhängigen Entwicklern genutzt, weil es flexible Einsatzmöglichkeiten bietet. Durch seine modulare Architektur und der aktiven Community wird FreeCAD kontinuierlich weiterentwickelt und verbessert.
\cite{FreeCAD}  \cite{FreeCAD_2}

\textbf{Vorteile}
\begin{itemize}
	\item Open Source Software
	\item Gratis
	\item Funktionsfähig für alle gängigen Betriebssysteme
	\item Aktive Community	 
\end{itemize}

\textbf{Nachteile}
\begin{itemize}
	\item weniger Nutzer als die bereits genannten Programme
	\item kein Industriestandard
	\item limitierte Funktionalitäten im Vergleich mit bezahl Software
\end{itemize}
\cite{FreeCADReviews}

\subsubsection{Autodesk Tinkercad}
Autodesk Tinkercad ist ein für Anfänger und Schüler entwickeltes Online-Tool. Wie der Name es schon verrät, ist es ein Produkt von Autodesk, wie das vorher genannte Autodesk Fusion 360. Dieses webbasierte Programm erfüllt alle grundlegenden Anforderungen und bietet intuitive Werkzeuge für Schüler, Lehrer und Privatpersonen. Zudem wird Tinkercad von vielen Bildungseinrichtungen verwendet, weil es leicht zugänglich und einfach zu erlernen ist. Dank seiner klaren Benutzeroberfläche und umfangreichen Tutorials ermöglicht Tinkercad einen schnellen Einstieg in die Welt des 3D-Designs und der Elektronik. 
\cite{Tinkercad}

\textbf{Vorteile}
\begin{itemize}
	\item Gratis
	\item Benutzerfreundliche UI
	\item intuitives Design
	\item Online Tool - kein Download 	
	\item viele einsteigerfreundliche Tutorials 
\end{itemize}

\textbf{Nachteile}
\begin{itemize}
	\item Weniger professionell
	\item kein Industriestandard 
\end{itemize}
\cite{TinkercadReviews}

\section{Mischpult}
 Für StageControl benötigen wir zur Ansteuerung einer Tonanlage ein Mischpult, um den von uns gewünschten Stereoeffekt zu erzeugen. Mischpulte, auch Mixer genannt, gibt es in ganz unterschiedlichen Größen. Die Größe hängt meist von der Anzahl der Kanäle ab, die das Mischpult bietet. Mischpulte werden immer dann verwendet, wenn mehrere Audiosignale verarbeitet werden müssen, wie auf Bühnen oder in Proberäumen. Bei allen Mischpulten gibt es eine sogenannte Leserichtung, die den Aufbau des Mischpults beschreibt. Von links nach rechts findet man die einzelnen Kanäle, und von oben nach unten die Bearbeitungsmöglichkeiten des Tons. Dazu kommt noch der sogannte Masterbereich. Hier wird alles gesteuert das Zentral gereglt werden muss, diese Einstellungen gelten dann für alle Kanäle.
\cite{Mischpult_Information}  \cite{Mischpult_Master}

\subsection{Wie funktioniert ein Mischpult}
Im Grunde summiert ein Mischpult alle Tonquellen zu einen einzigen Signal. So kann dann diese Signal über die Tonanlage ausgegeben werden. Bei Mischpulten gibt es eine Vielzahl von Knöpfen und Regler für jeden Kanal, diese sind wie vorher schon Erwähnt zur Bearbeitung des Tones da, im Rahmen dieser Diplomarbeit wird nicht auf dies genauer eingagangen.
\cite{Mischpult_Erklärung}

\subsection{Einsatzmöglichkeiten}
Mischpulte finden eine Vielzahl von Anwendungsmöglichkeiten, wie z.b. auf Bühnen oder in Proberäumen. Mischpulte übernehmen eine bandbreite von Aufgaben. Die Folgenden gehören zu den wichtigsten:
\begin{itemize}
	\item Signalverstärkung
	\item Bearbeitung von Signalen
	\item Signale zum Aufnahmesystem schicken
\end{itemize}
Diese und viele mehr Aufgaben kann ein Mischpult übernehemen.
\cite{Mischpult_Verwendungszweck}

\subsection{Unterschied Digital/Analog}
\subsubsection{Digital}
Unter einem digitalen Mischpult versteht man ein Mischpult, das analoge Signale in digitale Signale umwandelt und diese Signale mittels Software bearbeitet. Um ein digitales Signal wieder über eine Tonanlage auszugeben, muss es erneut in ein analoges Signal umgewandelt werden. Dies führt wiederum zu Latenzzeiten bei der Ausgabe des Signals.
\subsubsection{Analog}
Im Gegensatz dazu steht das analoge Mischpult. Es wandelt die Signale nicht in digitale um, sondern behält deren ursprüngliche analoge Form bei und bearbeitet sie ohne Software weiter. Ohne diese Umwandlung gibt es nahezu keine Latenzzeit bei der Ausgabe der Signale.
\cite{Mischpult_Analog/Digital}

\subsection{Anforderungen für StageControl}
Für StageControl ist es besonders wichitg das dass verwendete Mischpult entweder jeden Kanal einzeln pannen kann oder das der Master mindestens zwei Lautsprecher einer Tonanlage einzeln ansteuern kann. Somit kann der von uns gewünschte Stereoeffekt erzeugt werden.

\subsection{Vergleich 2er Modelle}

\section{Lichtsteuerung}
\subsection{Lichtanlagen bei StageControl}
\subsection{Ansteuerung von Lichtanlagen}

\section{Servomotoren}
\subsection{Was ist ein Servomotor}
\subsection{Einsatz von Servos bei StageCotrol}
\subsection{Modelle}
\subsection{Software zur Ansteuerung}


\section{Android App}
Um StageControl zu verwenden, benötigt man nicht nur die entsprechende Hardware, sondern auch die entsprechende Software (Android App). Um diese besagte Software zu entwickeln, benötigt man verschiedene Programme, die in einem weiteren Kapitel gegenübergestellt werden.

\subsection{Möglichkeiten der Umsetzung einer Android App}
Es gibt verschiedene Möglichkeiten der Umsetzung einer Android App. Folgende Programmiersprachen werden verwendet: Java, Kotlin oder auch JavaScript. Um diese Programmiersprachen zu verwenden, braucht man ein Framework oder eine Entwicklungsumgebung darunter fallen z.B. Android Studio, React Native, Flutter.

\section{Frameworks und Entwicklungumgebungen}
\subsection{Android Studio}
Android Studio ist nicht wirklich ein Framework, es handelt sich mehr um eine Entwicklungsumgebung, welche auf IntelliJ IEDA Community Edition zurückgreift und von Google entwickelt wurde. Die Hauptprogrammiersprachen sind Java und Kotlin. \cite{Android Studio}

\subsection{React Native}
React Native ist eine Open-Source-Software, welche von Facebook entwickelt worden ist. Es wurde entwickelt, um Entwicklern mit Erfahrungen in React die Möglichkeit zu geben, einfacher Android Apps basierend auf React zu entwicklen. Die Hauptprogrammiersprachen sind JavaScript oder TypeScript. \cite{React Native}

\subsection{Flutter}
Flutter ist eine Open-Source-Software, welche von Google entwickelt worden ist. Der Vorteil dieses Frameworks ist, dass man mit einem Programmcode in Flutter mehreren Apps auf verschiedenen Plattformen (Android, iOS) erstellen kann. Die Hauptprogrammiersprache ist Dart. \cite{Flutter}


\chapter{Ergebnisdokumentation}
\textbf{TBD}


