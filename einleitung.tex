\chapter{Einleitung}
\section{Ausgangslage}
Aktuell werden Ton- und Licht-Anlagen manuell gesteuert, um den gewünschten Stereo-Effekt und die Verfolgung per Spotlight der Künstler zu erreichen. Mit unserem Projekt wollen wir diese manuelle Arbeitsweise obsolet machen.

\section{Problemstellung}
Die Steuerung von Ton- und Lichtanlagen kann durch den Benutzer fehlerhaft werden. Dies erfolgt durch die manuelle Eingabe des Technikers. Die Koordinierung von Ton- und Lichtanlagen auf großen Bühnen kann zu einen unvorhersehbaren Arbeitsaufwand führen.

Doch es gibt ein Problem: Die Steuerung von Ton- und Lichtanlagen werden derzeit nur manuell gesteuert. Eine automatisierte Lösung gibt es noch nicht. 

\section{Persönlicher Standpunkt}
Wir hatten das Ziel eine Software zu Hardware Lösung zu entwickeln, die Ton- und Lichtanlagen automatisiert. Zudem sollte das Projekt in der Realität einen Nutzen haben.  

\section{Individuelle Themenstellung}
\textbf{Michael Becksteiner} setzt sich mit der Android App mit entsprechendem UI für den Bediener zu erstellen, um eine visuelle Darstellung der Bühne bzw. des Lichtes zu bieten auseinander. 

\textbf{Leo Pirringer} ist verantwortlich für die Umsetzung der erforderlichen Hardware und Implementierung dieser an der Schnittstelle zwischen Steuerungsgerät und Mischpult. 

\textbf{Carina Pospichal} beschäftigt sich mit der Umsetzung der Kommunikation zwischen der Hard- und Software. Sowie dem Logodesign und der Flyer Gestaltung. 

\section{Vorgehensweise} 
Das Kapitel 2 Grundlagen und Methoden beinhaltet die relevanten Grundlagen, Methoden und Technologien des Projekts. Es wird ein Gerüst mittels 3D-Drucker gedruckt auf welches der Servo gesteckt wird, welcher für die Steuerung der Schieberegler verantwortlich sein wird. Weiters wird in diesem Kapitel die Auswahl der Technologien, die für die Softwareanwendung, des Backend und des Frontend verwendet wird geschildert.  

Das Kapitel 3 Inhaltsdokumentation beinhaltet die Umsetzung des Projekts mit Verwendung der Grundlagen und Methoden. Alle Programmausdrücke der Dokumentation sind aus der praktischen Arbeit.  

Im Kapitel Verfasserverzeichnis ist aufgeführt, welches Teammitglied welche Abschnitte verfasst hat. Zudem enthält der Anhang Grafiken und Tabellen zum Projektmanagement. Die Projektergebnisse, sowie die Dokumente des Projektmanagements findet man in dem beigefügten Speichermedium. Eine detaillierte Auflistung des Inhalts vom Projekt findet sich im Anhang im Kapitel Beigelegtes Speichermedium. 