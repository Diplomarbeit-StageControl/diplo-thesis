\chapter{Einleitung}
\section{Ausgangslage}
Aktuell werden Ton- und Licht-Anlagen auf Bühnen manuell von einem Techniker gesteuert. Das Ziel dieses Technikers ist, den gewünschten Stereo-Effekt  und die Verfolgung per Spotlight der Künstler zu erreichen. Mit unserem Projekt wollen wir diese manuelle Arbeitsweise obsolet machen.

\section{Problemstellung}
Die Steuerung von Ton- und Lichtanlagen auf großen Bühnen wird meist von mehreren Benutzer beeinflusst und kann dadurch einen unvorsehbaren hohen Arbeitsaufwand und Fehler hervorführen. Derzeit gibt es für diese potenziellen Probleme noch keine Lösung.

\section{Individuelle Themenstellung}
\textbf{Michael Becksteiner} setzt sich mit der Android App, einem entsprechendem User-Interface und der Standortermittlung auseinander. 

\textbf{Leo Pirringer} ist verantwortlich für die Umsetzung der erforderlichen Hardware und Implementierung dieser an der Schnittstelle zwischen Steuerungsgerät und Mischpult. 

\textbf{Carina Pospichal} beschäftigt sich mit der Umsetzung der Kommunikation zwischen der Hard- und Software. Sowie dem Logodesign und der Flyer Gestaltung. 

\section{Vorgehensweise} 
\textbf{TBD}